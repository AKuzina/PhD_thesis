\section{Conclusion}
% \ak{Plan of the section (~0.25 page)
% \begin{itemize}
%     \item Equivariant Prior outperforms competitors
%     \item Rotation equivarince is usefull even when there is no rotation in th ecompressed sensing: GD converges faster and to a better local minimum
% \end{itemize}
% }
We consider a generative compressed sensing problem with unknown orientation of the measurement signal. This new setup motivates the usage of new generative prior models, which are capable of producing rotated images. We propose an equivariant variational autoencoder and extend theoretical convergence guarantees to the case of unknown rotation and equivariant generative prior. We experimentally show that the proposed equivariant prior is better or comparable to other benchmarks in terms of reconstruction quality and provide potential additional advantages in terms of convergence speed, as it usually requires fewer iterations (gradient descent steps).
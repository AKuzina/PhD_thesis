%-----SECTION-----

\onecolumn
\section{Theory}\label{appendix:theory}
We consider an attack, which has an additive structure:
\begin{align}
    \rvx^a = \rvx^r + \varepsilon, \\
    \text{such that }\|\varepsilon\| \leq \delta,
\end{align}
where $\delta$ is a radius of the attack. 

In the vanilla VAE setup we will get the latent code by sampling from $q_{\phi}(\rvz|\rvx)$. With our approach, instead, we get a sample from the following distribution:
\begin{align}
    q^{(t)}(\rvz|\rvx) = \int Q^{(t)}(\rvz|\rvz_0) q_{\phi}(\rvz_0|\rvx) d\rvz_0,
\end{align}

where $Q^{(t)}(\rvz|\rvz_0)$ is a transition kernel of MCMC with the target distribution $\pi(\rvz) = p_{\theta}(\rvz|\rvx) \propto p_{\theta}(\rvx|\rvz)p(\rvz)$.


\paragraph{Lemma 1}
Consider true posterior distributions of the latent code $\rvz$ for a data point $\rvx$ and its corrupted version $\rvx^a$.  Assume also that $\ln p_{\theta}(\rvz|\rvx)$ is twice differentiable over $\rvx$ with continuous derivatives at the neighbourhood around $\rvx=\rvx^r$. 
Then the KL-divergence between these two posteriors could be expressed using the small $o$ notation of the radius of the attack, namely:
\begin{equation}
     \KL {p_{\theta}(\rvz|\rvx^r)} {p_{\theta}(\rvz|\rvx^a)} = o(\|\varepsilon\|).
\end{equation}
\textit{Proof}\\
Let us use definition of the KL-divergence:
\begin{align} \label{eq:kl_def}
     \KL{p_{\theta}(\rvz|\rvx^r)}{p_{\theta}(\rvz|\rvx^a)} &= \E_{p_{\theta}(\rvz|\rvx^r)} \ln \frac{p_{\theta}(\rvz|\rvx^r)}{p_{\theta}(\rvz|\rvx^a)}  %\mathrm{d} \rvz 
\end{align}

Let us introduce $\ln p_{\theta}(\rvz|\rvx) = g(\rvx, \rvz)$. Assume that this function is differentiable at $\rvx = \rvx^r$. Then, we can apply Taylor expansion to $g(\rvx, \rvz)$ in the point $\rvx^r$ which yields:
\begin{equation}
    g(\rvx, \rvz) =  g(\rvx^r, \rvz) + (\rvx - \rvx^r)^T \nabla_{\rvx} g(\rvx, \rvz) \Big|_{\rvx^r} + R_1(\rvx, \rvx^r) .
\end{equation}

The remainder term in the Lagrange form can be written as
\begin{equation}\label{eq:remainder}
    R_1(\rvx, \rvx^r) = \tfrac{1}{2}(\rvx - \rvx^r)^T \nabla^2_{\rvx\rvx} g(\rvx + \theta (\rvx - \rvx^r), \rvz) (\rvx - \rvx^r), \, \theta \in (0, 1)
\end{equation}
Under the assumption that $g$ is twice differentiable with the continuous derivatives on the segment around $\rvx = \rvx^r$ the remainder term asymptotically converges to zero with $\rvx \rightarrow \rvx^r$. 
\begin{equation}
    R_1(\rvx, \rvx^r) = o(\|\rvx - \rvx^r\|).
\end{equation}

Then, the log-ratio of two distributions is the following:
\begin{align}
    &\ln \frac{p_{\theta}(\rvz| \rvx^r)}{p_{\theta}(\rvz|\rvx^a)}
    = g(\rvx^r, \rvz) - g(\rvx^a, \rvz) \\
    &\quad\quad\quad = g(\rvx^r, \rvz) - \left( g(\rvx^r, \rvz) + (\rvx^a - \rvx^r)^T \nabla_{\rvx} g(\rvx, \rvz) \Big|_{\rvx^r} + o(\|\rvx^a - \rvx^r\|)\right). \\
    &\quad\quad\quad =  -\varepsilon^T\nabla_{\rvx} g(\rvx, \rvz) \Big|_{\rvx^r} + o(\|\varepsilon\|) .
\end{align}
Notice that $\varepsilon^T \nabla_{\rvx} g(\rvx, \rvz) \Big|_{\rvx^r} $ is the dot product between $\varepsilon$ and $\nabla_{\rvx} g(\rvx, \rvz) \Big|_{\rvx^r} $, i.e., $\varepsilon^T \nabla_{\rvx} g(\rvx, \rvz) \Big|_{\rvx^r}  = \langle \varepsilon, \nabla_{\rvx} g(\rvx, \rvz) \Big|_{\rvx^r}  \rangle$.

We can now plug this into the KL-divergence definition (Eq.~\ref{eq:kl_def}):
\begin{align}
     \KL{p_{\theta}(\rvz|\rvx^r)}{p_{\theta}(\rvz|\rvx^a)} &= \E_{p_{\theta}(\rvz|\rvx^r)} \left[ -\langle \varepsilon, \nabla_{\rvx} \ln p_{\theta}(\rvz|\rvx) \Big|_{\rvx^r}  \rangle + o(\|\varepsilon\|)\right] \\
     &=   -\langle \varepsilon, \underbrace{ \E_{p_{\theta}(\rvz|\rvx^r)}  \nabla_{\rvx} \ln p_{\theta}(\rvz|\rvx) \Big|_{\rvx^r}}_{\text{A}(\rvx^r)}  \rangle + o(\|\varepsilon\|) \label{eq:kl_in_lemma1}
\end{align}

Note that for Eq.~\ref{eq:kl_in_lemma1} to hold we need to make sure that $\E_z R_1 = o(\|\varepsilon\|)$. As follows from Eq.~\ref{eq:remainder}, this requirement is satisfied if $\E_{z}\nabla^2_{\rvx\rvx} g(\rvx + \theta (\rvx - \rvx^r), \rvz)$ is bounded around $\rvx = \rvx^r$.

Let us take a closer to look at the term $\text{A}(\rvx^r)$ in the equation above:

\begin{align}
\text{A}(\rvx^r) &=    \E_{p_{\theta}(\rvz|\rvx^r)}  \nabla_{\rvx} \ln p_{\theta}(\rvz|\rvx) \Big|_{\rvx^r} \\
&=  \E_{p_{\theta}(\rvz|\rvx^r)}  \nabla_{\rvx} \ln \frac{p_{\theta}(\rvx|\rvz)p(\rvz)}{p_{\theta}(\rvx)} \Big|_{\rvx^r}   \label{eq:bayes_rule}  \\
&= \E_{p_{\theta}(\rvz|\rvx^r)}  \nabla_{\rvx} \ln p_{\theta}(\rvx|\rvz) \Big|_{\rvx^r} - \E_{p_{\theta}(\rvz|\rvx^r)}  \nabla_{\rvx} \ln p_{\theta}(\rvx) \Big|_{\rvx^r} \\
&= \int p_{\theta}(\rvz|\rvx^r)\frac{\nabla_{\rvx} p_{\theta}(\rvx|\rvz)\Big|_{\rvx^r}}{p_{\theta}(\rvx^r|\rvz)}d\rvz - \int p_{\theta}(\rvz|\rvx^r)\frac{\nabla_{\rvx} p_{\theta}(\rvx)\Big|_{\rvx^r}}{p_{\theta}(\rvx^r)} d\rvz  
\label{eq:log_derivative}\\
&= \int \frac{p_{\theta}(\rvz)}{p_{\theta}(\rvx^r)} \nabla_{\rvx} p_{\theta}(\rvx|\rvz)\Big|_{\rvx^r} d\rvz - \frac{\nabla_{\rvx} p_{\theta}(\rvx)\Big|_{\rvx^r}}{p_{\theta}(\rvx^r)}  \underbrace{\int p_{\theta}(\rvz|\rvx^r)d\rvz }_{=1}  \label{eq:bayes_rule_2} \\
&= \frac{1}{p_{\theta}(\rvx^r)} \left[ \int p(\rvz) \nabla_{\rvx} p_{\theta}(\rvx|\rvz)\Big|_{\rvx^r} d\rvz - \nabla_{\rvx} p_{\theta}(\rvx)\Big|_{\rvx^r} \right]\\
&= \frac{1}{p_{\theta}(\rvx^r)} \left[ \E_{p(\rvz)} \nabla_{\rvx} p_{\theta}(\rvx|\rvz)\Big|_{\rvx^r} - \nabla_{\rvx} \E_{p(\rvz)}p_{\theta}(\rvx|\rvz)\Big|_{\rvx^r} \right]\\
&= \frac{1}{p_{\theta}(\rvx^r)}  \underbrace{\left[ \E_{p(\rvz)} \nabla_{\rvx} p_{\theta}(\rvx|\rvz)\Big|_{\rvx^r} -  \E_{p(\rvz)}\nabla_{\rvx} p_{\theta}(\rvx|\rvz)\Big|_{\rvx^r} \right]}_{=0} = 0.
\end{align}

where we use Bayes rule in Eq.~\ref{eq:bayes_rule}, Eq.~\ref{eq:bayes_rule_2}, log-derivative trick in Eq.~\ref{eq:log_derivative}. 

We have shown that $\text{A}(\rvx^r) = 0$, therefore, from Eq.~\ref{eq:kl_in_lemma1} we have:
 
\begin{align}
    \KL{p_{\theta}(\rvz|\rvx^r)}{p_{\theta}(\rvz|\rvx^a)} &= -\langle \varepsilon, \text{A}(\rvx^r) \rangle + o(\|\varepsilon\|) \\
    &= o(\|\varepsilon\|). 
\end{align}
% \begin{align}
%     \KL{p_{\theta}(\rvz|\rvx^r)}{p_{\theta}(\rvz|\rvx^a)} &= - \frac{\langle \varepsilon, \E_{p(\rvz)}\nabla_{\rvx^r} p_{\theta}(\rvx^r|\rvz)\rangle }{p_{\theta}(\rvx^r)} + o(\|\varepsilon\|) +
%     \frac{\langle \varepsilon, \E_{p(\rvz)} \nabla_{\rvx^r} p_{\theta}(\rvx^r|\rvz) \rangle }{p_{\theta}(\rvx^r)} + o(\|\varepsilon\|) \\
%     &= o(\|\varepsilon\|) .
% \end{align}
$\hfill\blacksquare$

%%%%%%%%%%%%%%%%%%%%%%%%%%%%%%%%%%%%%%%%%%%%%%%%%%%%%%%%
\paragraph{Lemma 2}
The Total Variation distance (TV) between the variational posterior with MCMC for a given corrupted point $\rvx^a$, $q^{(t)}(\rvz|\rvx^a)$, and the variational posterior for a given data point $\rvx^r$, $q_{\phi}(\rvz|\rvx^r)$, can be upper bounded by the sum of the following three components:
\begin{equation}
    \begin{aligned}
\TV{q^{(t)}(\rvz|\rvx^a)}{q_{\phi}(\rvz|\rvx^r)} &\leq 
    \TV{q^{(t)}(\rvz|\rvx^a)}{p_{\theta}(\rvz|\rvx^a)} \\
     &\quad+ 
    \sqrt{\tfrac12 \KL{p_{\theta}(\rvz|\rvx^r)}{p_{\theta}(\rvz|\rvx^a)}  } \\
    &\quad+
   \sqrt{\tfrac12  \KL{q_{\phi}(\rvz|\rvx^r)}{p_{\theta}(\rvz|\rvx^r)}}.
    \end{aligned}
\end{equation}
\textit{Proof}\\
Total variation is a proper distance, thus, the triangular inequality holds for it. For the proof, we apply the triangular inequality twice. First, we use the triangle inequality for $\text{TV}\left[ q^{(t)}(\rvz|\rvx^a), q_{\phi}(\rvz|\rvx^r) \right]$, namely: 
\begin{equation}
    \begin{aligned}
    \TV{q^{(t)}(\rvz|\rvx^a)}{q_{\phi}(\rvz|\rvx^r)} &\leq 
    \TV{q^{(t)}(\rvz|\rvx^a)}{p_{\theta}(\rvz|\rvx^r)}  \\
    &\quad+ \TV{p_{\theta}(\rvz|\rvx^r)}{q_{\phi}(\rvz|\rvx^r)}.        
    \end{aligned}
\end{equation}

Second, we utilize the triangle inequality for $\TV{q^{(t)}(\rvz|\rvx^a)} {p_{\theta}(\rvz|\rvx^r) }$, that is:
\begin{equation}
    \begin{aligned}
    \TV{q^{(t)}(\rvz|\rvx^a)}{p_{\theta}(\rvz|\rvx^r)}
    &\leq 
    \TV{q^{(t)}(\rvz|\rvx^a)}{p_{\theta}(\rvz|\rvx^a)}  \\
     &\quad+ \TV{p_{\theta}(\rvz|\rvx^a)}{p_{\theta}(\rvz|\rvx^r)} .
    \end{aligned}
\end{equation}

Combining the two gives us the following upper bound on the initial total variation:

\begin{align}
\TV{q^{(t)}(\rvz|\rvx^a)}{q_{\phi}(\rvz|\rvx^r)}
&\leq 
    \TV{q^{(t)}(\rvz|\rvx^a)}{p_{\theta}(\rvz|\rvx^a)} \\
     &\quad+ 
    \TV{p_{\theta}(\rvz|\rvx^a)}{p_{\theta}(\rvz|\rvx^r)} \label{eq:three_comp_2}\\
    &\quad+
    \TV{p_{\theta}(\rvz|\rvx^r)}{q_{\phi}(\rvz|\rvx^r)} \label{eq:three_comp_3}
\end{align}

Moreover, the Total Variation distance is a lower bound of the KL-divergence (by Pinsker inequality):
\begin{equation}
    \TV{p(\rvx)}{q(\rvx)}
    \leq \sqrt{\tfrac12 \KL{p(\rvx)}{q(\rvx)}} .
\end{equation}

Applying Pinsker inequality to 
% the second and the third terms in the right-hand part of \eqref{eq:three_comp} 
Eq.~\ref{eq:three_comp_2} and Eq.~\ref{eq:three_comp_3}
yields:
\begin{align}
\TV{q^{(t)}(\rvz|\rvx^a)}{q_{\phi}(\rvz|\rvx^r)}
&\leq 
% \underbrace{
    \TV{q^{(t)}(\rvz|\rvx^a)}{p_{\theta}(\rvz|\rvx^a)} \\
% }_{\substack{\text{1)}\stackrel{t}{\rightarrow}  0}}
     &\quad+ 
% \underbrace{
    \sqrt{\tfrac12 \KL{p_{\theta}(\rvz|\rvx^r)}{p_{\theta}(\rvz|\rvx^a)}  } \\
% }_{\substack{\text{2) Similarities of posteriors}}}
    &\quad+
% \underbrace{
   \sqrt{\tfrac12  \KL{q_{\phi}(\rvz|\rvx^r)}{p_{\theta}(\rvz|\rvx^r)}} .
% }_{\substack{\text{3) Posterior gap}}}
\end{align}
$\hfill\blacksquare$

%%%%%%%%%%%%%%%%%%%%%%%%%%%%%%%%%%%%%%%%%%%%%%%%%%%%%%%%
\paragraph{Theorem 1}
The upper bound on the total variation distance between samples from MCMC for a given corrupted point $\rvx^a$, $q^{(t)}(\rvz|\rvx^a)$, and the variational posterior for the given real point $\rvx$, $q_{\phi}(\rvz|\rvx)$, is the following:

\begin{equation}
\begin{aligned}
    \TV{q^{(t)}(\rvz|\rvx^a)}{q_{\phi}(\rvz|\rvx^r)}
&\leq 
\TV{q^{(t)}(\rvz|\rvx^a)}{p_{\theta}(\rvz|\rvx^a)}
      \\ 
 &\quad+   \sqrt{\tfrac12  \KL{q_{\phi}(\rvz|\rvx^r)}{p_{\theta}(\rvz|\rvx^r)}} \\
  &\quad+ o(\sqrt{\|\varepsilon\|}).
    \end{aligned}
\end{equation}
\textit{Proof}\\
Combining \textbf{Lemma 1} and \textbf{2} we get:
\begin{align}
    \TV{q^{(t)}(\rvz|\rvx^a)}{q_{\phi}(\rvz|\rvx^r)}
\underbrace{\leq }_{\text{Lemma 2}}
    &\TV{q^{(t)}(\rvz|\rvx^a)}{p_{\theta}(\rvz|\rvx^a)} \\
     &+ 
    \sqrt{\tfrac12 \KL{p_{\theta}(\rvz|\rvx^r)}{p_{\theta}(\rvz|\rvx^a)}} \\
    &+
   \sqrt{\tfrac12  \KL{q_{\phi}(\rvz|\rvx^r)}{p_{\theta}(\rvz|\rvx^r)}} \\
   \underbrace{=}_{\substack{\text{Lemma 1}}}
 &\TV{q^{(t)}(\rvz|\rvx^a)}{p_{\theta}(\rvz|\rvx^a)}\\
     &+ 
    \sqrt{\tfrac12 o(\|\varepsilon\|)  }\\
    &+
   \sqrt{\tfrac12  \KL{q_{\phi}(\rvz|\rvx^r)}{p_{\theta}(\rvz|\rvx^r)}}.
\end{align}

Note that $\sqrt{\tfrac12 o(\|\varepsilon\|)  } =  o(\sqrt{\|\varepsilon\|}) $ that gives us the final expression:

\begin{equation}
\begin{aligned}
    \TV{q^{(t)}(\rvz|\rvx^a)}{q_{\phi}(\rvz|\rvx^r)}
&\leq 
 \TV{q^{(t)}(\rvz|\rvx^a)}{p_{\theta}(\rvz|\rvx^a)}
     \\
  &\quad+ \sqrt{\tfrac12  \KL{q_{\phi}(\rvz|\rvx^r)}{p_{\theta}(\rvz|\rvx^r)}} \\
    &\quad+ o(\sqrt{\|\varepsilon\|}).
    \end{aligned}
\end{equation}
$\hfill\blacksquare$



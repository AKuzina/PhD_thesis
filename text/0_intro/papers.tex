\chapter*{List of Publications}\label{chapter:papers}
\addcontentsline{toc}{chapter}{\nameref{chapter:papers}}

The following publications form the basis of this thesis:
\begin{itemize}
    \item \textbf{Kuzina, A.}\footnote[1]{Shared first authorship}, Egorov, E.\footnotemark[1], Burnaev, E. (2021). BooVAE: Boosting approach for continual learning of VAE. Advances in Neural Information Processing Systems (NeurIPS).
    \item Deja, K\footnotemark[1]., \textbf{Kuzina, A.}\footnotemark[1], Trzcinski, T., Tomczak, J. (2022). On analyzing generative and denoising capabilities of diffusion-based deep generative models. Advances in Neural Information Processing Systems (NeurIPS).
    \item \textbf{Kuzina, A.}, Welling, M., Tomczak, J. M. (2021). Diagnosing vulnerability of variational auto-encoders to adversarial attacks. In ICLR Workshop on Robust Machine Learning.
    \item \textbf{Kuzina, A.}, Welling, M., Tomczak, J. (2022). Alleviating adversarial attacks on variational autoencoders with MCMC. Advances in Neural Information Processing Systems (NeurIPS).
    \item \textbf{Kuzina, A.}, Pratik, K., Massoli, F. V., Behboodi, A. (2022). Equivariant priors for compressed sensing with unknown orientation. In International Conference on Machine Learning (ICML).
    \item \textbf{Kuzina, A.},  Tomczak, J. (2024). Hierarchical VAE with a Diffusion-based VampPrior.
\end{itemize}

As a first author I have contributed in all aspects (formulating problem, designing solution, running experiments and writing the text) to the publications listed above.
Two articles were written in equal contribution with other co-authors.  
In these articles, these co-authors also contributed substantially to the conception of the ideas or experiments.
Max Welling and Jakub Tomczak provided supervision, guidance, insight, and technical advice.

%In the two papers were first authorship is shared me and the other first author contributed equally to the 
% In "MDP Homomorphic Networks: Group Symmetries in Reinforcement Learning." Daniel Worrall proposed the Symmetrizer, which was implemented by us jointly.
% In "Contrastive Learning of Structured World Models." Thomas Kipf
% contributed in all aspects. I provided reinforcement learning insights,
% proposed the use of action factorization and the evaluation method,
% and ran baseline experiments. Figures and tables reproduced with permission. In "Geometric and Physical Quantities improve E(3) Equivariant Message Passing.", Johannes Brandstetter, Rob Hesselink, and Erik
% Bekkers contributed in all aspects. I provided the original architectural
% idea and took an advisory role. Figures reproduced with permission.

Other papers, that are not included in this thesis, but that I have contributed to during my PhD:
\begin{itemize}
    \item Romero, D. W., \textbf{Kuzina, A.}, Bekkers, E. J., Tomczak, J. M., Hoogendoorn, M. (2022). CKConv: Continuous Kernel Convolution For Sequential Data. In International Conference on Learning Representations (ICLR).
    \item Chen, H.\footnotemark[1], \textbf{Kuzina, A.}\footnotemark[1], Esmaeili, B., Tomczak, J. M. (2024). Variational Stochastic Gradient Descent for Deep Neural Networks. In ICML workshop on Advancing Neural Network Training (WANT)
    \item Zając, M., Deja, K., \textbf{Kuzina, A.}, Tomczak, J.M., Trzciński, T., Shkurti, F. and Miłoś, P. (2023). Exploring continual learning of diffusion models. In CVPR Workshop on Continual Learning in Computer Vision
\end{itemize}